%=====================================================
% Glossary
%=====================================================

%-----------------------------------------------------
% Samples
%-----------------------------------------------------

% Usage:
% \gls{glos:AD} is pretty interesting. If we have a cross reference from
% the acronyms, we can also directly go to that using \gls{AD}; this
% requires then that over there, we have something like
%  description=\glslink{glos:AD}{Active Directory}

% \newglossaryentry{glos:AD}{
% name=Active Directory,
% description={Active Directory is the directory service for
% Windows based networks, that allows central organization and
% administration of any network resource.
% It allows a single-sign-on concept independent from network
% topologies or network protocols. As a prerequisite you need
% a Windows Server acting as Domain Controller. This computer
% stores all necessary data, e.\,g.~usernames and corresponding
% passwords.}
% }


%-----------------------------------------------------
% Content
%-----------------------------------------------------


\setlength{\abovedisplayskip}{0pt}

%<content>%
% \renewcommand{\theHequation}{\theHsection.\equationgrouping\arabic{equation}}

%=================================================================================
% Break Even Point
%=================================================================================

\def\glosdescbep{
The Break Even Point, or BEP, is the point where through a cost volume analysis, the total revenue
equals the total costs: %\citep[304]{Atrill-2008qf}:
\begin{subequations}
\begin{align}
\text{Total Sales Revenue} &\overset{!}{=} \text{Total Costs} \label{eqn:bep}
\intertext{and, with}
\text{Total Sales Revenue} &= \text{Fixed Costs} + \text{Total Variable Costs}
\intertext{we get, for the number $n$ of produced units at BEP:}
n \times \text{Sales Revenue per Unit} &= \text{Fixed Costs} + n \times  \text{Variable Costs per Unit}
\intertext{and solving for $n$, we get the number of units to sell at BEP:}
\end{align}
\end{subequations}
\begin{equation}\label{eqn:bep_contribution}
\tikzmarkin{bep2}(1.0,-0.75)(-1.0,0.85)
        \tikz[baseline]{
            \node[fill=red!20,anchor=base] (t1)
            {$ n $};
        } =
        \frac{\tikz[baseline]{
           \node[fill=blue!10, anchor=base] (t2)
            {$\text{Fixed Costs}$};
        }}{
        \tikz[baseline]{
            \node[fill=green!20,anchor=base] (t3)
            {$\text{Sales Revenue per Unit} - \text{Variable Cost per Unit}$};
        }}
\tikzmarkend{bep2}
\end{equation}

\begin{tikzpicture}[overlay]
\coordinate (col-aa) at ($(bep2)+(2.5,0.4)$);
\coordinate (col-ab) at ($(bep2)+(5.5,-2.0)$);
\node[align=left,right] at (col-aa) {\small{\emph{Number of units produced at BEP}}};
\node[align=left,left] at (col-ab) {\small{\emph{Contribution}}};
        \path[red,-stealth,->] (col-aa) edge [bend right] (t1);
        \path[red,-stealth,->] (col-ab) edge [out=0, in=-90] (t3);
\end{tikzpicture}

\bigskip \emph{Note:} As the fixed costs are time based, the BEP must be expressed with regards to a period of time.
}
\newglossaryentry{glos:bep}{
  name=Break Even Point (BEP),
  description={\glosdescbep}
}


%</content>%























