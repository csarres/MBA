%=====================================================
% Document Type and Geometry
%=====================================================

\documentclass[12pt,a4paper,fleqn,twoside]{\doctype}


%=====================================================
% Some Helpers
%=====================================================

\usepackage{morewrites}								% More write space
\usepackage{lipsum}										% Lorem Impsum...
\usepackage{xparse}										% Parsing of Parameters
\usepackage{xifthen}									% If Then Else
\usepackage{ifpdf}										% Generating PDF or not
\ifpdf\usepackage{cmap}\fi						% Makes PDF searchable, for Skim
\usepackage{xstring}									% String functions
\usepackage[titles]{tocloft}					% New Listofs...
\usepackage{subfig}                   % side by side figures
\usepackage[export]{adjustbox}        % for alignment of subfloats
\usepackage{lastpage}

% The following three lines remove page numbers from the
% part page and chapter first page:
\usepackage{etoolbox}
\patchcmd{\chapter}{plain}{empty}{}{}
\patchcmd{\part}{plain}{empty}{}{}

%=====================================================
% Document Geometry
%=====================================================

\usepackage[hmarginratio=1:1]{geometry}
\geometry{
  left=2.2cm,													% left margin
  top=3.5cm,													% top  margin
  textwidth=16.6cm,										% width of text block
  textheight=21.0cm}									% height of text block 22
\setlength{\headheight}{1cm}					% height of header
\setlength{\headsep}{1cm}							% distance of header
\setlength{\footskip}{0.5cm}					% distance of footer
\renewcommand{\baselinestretch}{1.0}	% 1line distance
\clubpenalty10000											% Don't want clubss
\widowpenalty10000										% Don't want widows
\hbadness 10000												% Reduce underfull hbox warnings
\sloppy															  % Allow sloppy layout, reduce hyphenations
\raggedbottom\topskip=10pt

%=====================================================
% Language
%=====================================================

\usepackage[russian,american]{babel}	% Internationalization
\usepackage[utf8]{inputenc}						% Allow for utf-8 in input files
\usepackage[OT2,T1]{fontenc}					% Breaks utf-8 in bibliography
\usepackage{uarial}
\usepackage{csquotes}  								% Extended quoting
\selectlanguage{american}							% Select English


%=====================================================
% Fixups for HTML
%=====================================================

\ifthenelse{\equal{\doctype}{book}}{
\ifpdf
\else
  \renewcommand\markboth{\null}			  % Markboth makes no sense in HTML
\fi
}{}

\ifpdf
	\usepackage[english]{varioref}			% vref and related
\else
	\usepackage{tex4ht}
   \newcommand\vref{\ref}							% varioref doesn't work in HTML
   \renewcommand\textwidth{\linewidth}
\fi


%=====================================================
% Word counting
%=====================================================

\ifpdf
\else
  \def\HCode#1{}											% Allow plain HTML output
\fi
\newcommand\wcounta{\ifpdf\else\HCode{<!-- COUNT -->}\fi}
\newcommand\wcounte{\ifpdf\else\HCode{<!-- /COUNT -->}\fi }


%=====================================================
% Headers
%=====================================================

\usepackage{fancyhdr}									% Fancy Page Headers
\usepackage[Sonny]{fncychap}					% Fancy Chapter Headers
\ChNumVar{\fontsize{60}{62}}					% Set Font Size for Chapter Headers

\usepackage{epigraph}
\setlength{\epigraphwidth}{0.5\linewidth}
\setlength{\epigraphrule}{0.1pt}
%\renewcommand*{\textflush}{flushright}
\renewcommand*{\epigraphsize}{\normalsize\itshape}
\let\Epigraph\epigraph
\renewcommand\epigraph[2]{\Epigraph{\slshape\singlespacing ``#1''}{\footnotesize\textsc{#2}}\par}
% %Change width as follows:
% %\newlength{\savedepigraphwidth}
% %\setlength{\savedepigraphwidth}{\epigraphwidth}
% %\setlength{\epigraphwidth}{0.8\linewidth}

%=====================================================
% Floats, Tables, Equations, etc.
%=====================================================

\usepackage{boxedminipage}						% Minipage with a box around it
\usepackage{wrapfig}									% Allow for floats wrapped by text
\usepackage{caption}						% captions with hanging indent
\usepackage{capt-of} 									% caption{figure}[]{}, outside float
\usepackage[section,above] {placeins}	% Prevent floats floating past the section
\usepackage{float}										% Improved control over floats
\usepackage{booktabs}									% Better tables
\usepackage{threeparttablex}  				% Footnotes in tables
\usepackage{multicol}                 % Multiple Columns
\usepackage{multirow}                 % Multiple Rows
\usepackage{adjustbox}								% Flexible boxes, color, rotate, etc.
\usepackage{rotating}                 % Landscape Figures
\usepackage{fancybox}									% Boxes, also for equations
\usepackage{array}										% Better tabular and array environments
\usepackage{colortbl}									% Colored Tables
\usepackage{color}										% Color functionality
\usepackage[table]{xcolor}						% Extended color functionality (blue!20 etc.)
\usepackage{marginnote}
\usepackage[skip=10pt plus1pt, indent=0pt]{parskip}
%\usepackage[parfill]{parskip}
\usepackage[titletoc,title]{appendix}

%
% Alter some LaTeX defaults for better treatment of figures:
% See p.105 of "TeX Unbound" for suggested values.
% See pp. 199-200 of Lamport's "LaTeX" book for details.
%
% General parameters, for ALL pages:
%
\renewcommand{\topfraction}{0.9}			% max fraction of floats at top
\renewcommand{\bottomfraction}{0.8}		% max fraction of floats at bottom
%
% Parameters for TEXT pages (not float pages):
%
\setcounter{topnumber}{2}							% max floats at top of page
\setcounter{bottomnumber}{2}					% max floats at bottom of page
\setcounter{totalnumber}{4}     			% max floats on page
\setcounter{dbltopnumber}{2}    			% for 2-column pages
\renewcommand{\dbltopfraction}{0.9}		% fit big float above 2-col. text
\renewcommand{\textfraction}{0.07}		% allow minimal text w. figs
%
% Parameters for FLOAT pages (not text pages):
%
\renewcommand{\floatpagefraction}{0.7}	 % require fuller float pages
% N.B.: floatpagefraction MUST be less than topfraction !!
\renewcommand{\dblfloatpagefraction}{0.7}% require fuller float pages
% Remember to use [htp] or [htpb] for placement

%=====================================================
% Code Listings
%=====================================================

\usepackage{listings}
\definecolor{lstemph}{rgb}{0,0.39,0}
\definecolor{lstnumbers}{rgb}{0.59,0.57,0.43}
\definecolor{lstcomments}{rgb}{0.33,0.35,0.69}
\lstloadlanguages{Java,C++}
\lstset{language=Java,
        extendedchars=true,
        basicstyle=\ttfamily\tiny,
        keywordstyle=\color{lstnumbers},
        identifierstyle=\color{black},
        commentstyle=\color{lstcomments},
        stringstyle=\ttfamily\color{blue},
        showstringspaces=true,
        numbers=left,
        stepnumber=1,
        numberstyle=\tiny\ttfamily\color{lstnumbers},
        numbersep=12pt,
        frame=single,
        fontadjust=true,
        xleftmargin=3.5pt,
        xrightmargin=3.5pt,
        escapeinside={(*}{*)}}

%=====================================================
% Footnotes / Endnotes
%=====================================================

\usepackage{templates/sty/botfnote/botfnote}				% Force footnotes to the bottom
%\usepackage[flushmargin,hang]{footmisc}	% More footnote options
%\usepackage[backref,counter-format=arabic]{templates/sty/enotez/enotez} % Backreferencing Endnotes
\usepackage{enotez} % Backreferencing Endnotes
\setenotez{backref,counter-format=arabic}

\ifpdf
  \usepackage{templates/sty/hyperendnote/hyperendnote} %  Referenced Endnotes
\fi

\usepackage{setspace}

\DeclareInstance{enotez-list}{itemize}{list}{
  list-type = itemize,
  number = \enmark{#1} ,
  format = \footnotesize,
}

\renewcommand*{\thefootnote}{[\arabic{footnote}]}

\usepackage[flushmargin,hang]{footmisc}	% More footnote options

%=====================================================
% Hyperref
%=====================================================

\ifpdf
\usepackage[%pdftex,
            pdfpagemode={UseOutlines},
            pdfstartview={FitH},
            colorlinks=true,
            linkcolor={blue},
            citecolor={blue},
            urlcolor={blue},
            bookmarks=true,
            bookmarksopen=true,
            %hyperfootnotes=false,
            %pdfpagemode=FullScreen,
            %hyperindex=false,
            plainpages=false,
            %hypertexnames=false,
            pdfpagelabels]{hyperref}
\else
\usepackage[tex4ht]{hyperref}
\fi

\usepackage{footnotebackref}
\ifpdf\usepackage{tabu}\fi						% Nice tables



%=====================================================
% Load Macros
%=====================================================

%=====================================================
% Makros
%=====================================================


%-----------------------------------------------------
% Content Substitutions:
%-----------------------------------------------------


\def\saas{\gls{saas}\index{SaaS}}
\def\paas{\gls{paas}\index{PaaS}}
\def\iaas{\gls{iaas}\index{IaaS}}

%-----------------------------------------------------
% Simple Substitutions:
%-----------------------------------------------------

\def\ni{\noindent}
\def\usw{$[\dots]$}
\def\daher{$\rightarrow$}
\def\tab{\hspace{2 cm}}
\def\fn{\footnote}
\def\en{\endnote}
\def\Unterschrift{\newline \includegraphics[width=4cm]{fig/unter} \newline}
\def\unterschrift{\Unterschrift}
\newcommand{\bs}{$\backslash$}

%
% Mathematical proofs
%
\def\LRA{\Leftrightarrow\mkern40mu}
\def\RA{\Rightarrow\mkern40mu}
\newcommand{\qed}{\hfill \ensuremath{\hfill \blacksquare}}

%
% Currency Symbols
%
\def\gbpm{\text{\,M\pounds}}
\def\gbp{\text{\,\pounds}}
\def\usdm{\text{\,M\$}}
\def\usd{\text{\,\$}}
\def\eurm{\text{\,M\euro}}
\def\eur{\text{\,\euro}}
\def\perc{\text{~\%}}
\newcommand{\Rho}{\mathrm{P}}

%-----------------------------------------------------
% Safe Equals
%-----------------------------------------------------
\makeatletter
\newcommand{\sequals}[2]{%
  \ifnum\pdfstrcmp{#1}{#2}=\z@
    \expandafter\@firstoftwo
  \else
    \expandafter\@secondoftwo
  \fi}
\makeatother


%-----------------------------------------------------
% dangerous / ddangerous etc. environments a la Knuth:
%-----------------------------------------------------

\font\manual=manfnt
\def\dbend{{\manual\char127}}
\def\d@nger{\medbreak\begingroup\clubpenalty=10000
  \def\par{\endgraf\endgroup\medbreak}
\noindent\hang\hangafter=-2  \hbox
to0pt{\hskip-\hangindent\dbend\hfill}\ninepoint}
\outer\def\danger{\d@nger}
\def\dd@nger{\medbreak\begingroup\clubpenalty=10000
  \def\par{\endgraf\endgroup\medbreak}
\noindent\hang\hangafter=-2  \hbox
to0pt{\hskip-
\hangindent\dbend\kern1pt\dbend\hfill}\ninepoint}
\outer\def\ddanger{\dd@nger}
\def\enddanger{\endgraf\endgroup}

%
% bulletpoint environment:
%
% First argument is the content of the bullet. Default: ?
% Second argument is the vertical alignment. Default: centered
% Third argument is no if not to run a circle. Default: Yes
%
\newsavebox{\fmbox}
\makeatletter
\NewDocumentEnvironment{bulletpoint}{O{?} O{c} O{yes}}%
{
 \ifthenelse{\isempty{#1}}{\def\m@sign{?}}{\def\m@sign{#1}}%
 \ifthenelse{\isempty{#2}}{\def\m@align{c}}{\def\m@align{#2}}%
 \begin{lrbox}{\fmbox}
 \begin{minipage}[\m@align]{0.2cm}~\end{minipage}
 \ifthenelse{\equal{#3}{yes}}{
   \begin{minipage}[\m@align]{1.5cm}\hspace{\fill}\circled{\m@sign}\end{minipage}
 }{
   \begin{minipage}[\m@align]{1.5cm}\hspace{\fill}\m@sign\end{minipage}
 }
 \vrule width 2pt \begin{minipage}[\m@align]{0.1cm}~\end{minipage}
 \begin{minipage}[\m@align]{12.8cm}
}
{\end{minipage}\end{lrbox}\usebox{\fmbox}
}
\makeatother
\newenvironment{handright}%
{
 DEPRECATED! USE bulletpoint with ding(45)
 \begin{lrbox}{\fmbox}
 \begin{minipage}[t]{0.2cm}~\end{minipage}
 \begin{minipage}[t]{1.5cm}\hspace{\fill}\ding{45}\end{minipage}
 \vrule width 2pt \begin{minipage}[t]{0.1cm}~\end{minipage}
 \begin{minipage}[t]{10.8cm}
}
{\end{minipage}\end{lrbox}\usebox{\fmbox}
}
\newenvironment{notdangerous}%
{
 \begin{lrbox}{\fmbox}
 \begin{minipage}[t]{1cm}~\end{minipage}
 \begin{minipage}[t]{1.5cm}\hspace{\fill}~\end{minipage}
 \begin{minipage}[t]{0.1cm}~\end{minipage}
 \begin{minipage}[t]{12.5cm}
}
{\end{minipage}\end{lrbox}\usebox{\fmbox}
}
\newenvironment{dangerous}%
{
 \begin{lrbox}{\fmbox}
 \begin{minipage}[t]{0.2cm}~\end{minipage}
 \begin{minipage}[t]{1.5cm}\hspace{\fill}\dbend\end{minipage}
 \begin{minipage}[t]{0.1cm}~\end{minipage}
 \begin{minipage}[t]{10.8cm}
}
{\end{minipage}\end{lrbox}\usebox{\fmbox}
}
\newenvironment{ddangerous}%
{
 \begin{lrbox}{\fmbox}
 \begin{minipage}[t]{0.2cm}~\end{minipage}
 \begin{minipage}[t]{1.5cm}\hspace{\fill}\dbend\dbend\end{minipage}
 \begin{minipage}[t]{0.1cm}~\end{minipage}
 \begin{minipage}[t]{10.8cm}
}
{\end{minipage}\end{lrbox}\usebox{\fmbox}
}
\newenvironment{dddangerous}%
{
 \begin{lrbox}{\fmbox}
 \begin{minipage}[t]{0.2cm}~\end{minipage}
 \begin{minipage}[t]{1.5cm}\hspace{\fill}\dbend\dbend\dbend\end{minipage}
 \begin{minipage}[t]{0.1cm}~\end{minipage}
 \begin{minipage}[t]{10.8cm}
}
{\end{minipage}\end{lrbox}\usebox{\fmbox}
}

%-----------------------------------------------------
% Other environments:
%-----------------------------------------------------

\newcommand{\pfooter} {
\vspace*{0.2cm}
\setlength{\tabcolsep}{0.05cm}
\tiny
\centerline{
\begin{tabular}{p{1.5cm}p{4.5cm}p{6.6cm}p{0.5cm}p{2.96cm}}
\toprule[0.25pt]
\makebox[1.50cm][l]{\tfooterleftlabelup:} &
\makebox[4.50cm][l]{\tfooterleftup}&
\makebox[6.00cm][c]{\tfootermiddleup}&
\makebox[0.50cm][l]{$\:$\tfooterrightlabelup:}&
\makebox[2.96cm][r]{\tfooterrightup}
\\%\midrule[0.15pt]
\makebox[1.50cm][l]{\tfooterleftlabeldown:} &
\makebox[4.50cm][l]{\tfooterleftdown}&
%\makebox[6.00cm][c]{\tsubtitlecover\ (\tsubsubtitlecover)}&
\makebox[6.00cm][c]{\tfootermiddledown}&
\makebox[0.50cm][l]{$\:$\tfooterrightlabeldown:}&
\makebox[2.96cm][r]{\tfooterrightdown}
\\
\bottomrule[0.25pt]
\end {tabular}
}}

\renewcommand*\copyright{{\usefont{T1}{lmr}{m}{n}\raisebox{1pt}{\textcopyright}}}
\definecolor{pfhdr}{rgb}{0.17255, 0.18824, 0.3529}
\definecolor{white}{rgb}{1, 1, 1}




%-----------------------------------------------------
% Exhibit:
%
% Use:
%
% \begin{exhibit}[TOCCaption]{Caption}[rRiIlLoO][.5][32]
%
% Parameters:
%
% 1. Optional parameter TOCCaption gives the
%     entry into \listofexhibit
%
% 2. Mandatory parameter: Caption
%
% 3. Optional parameter:  Lines to wrap around.
%     Not needed unless positioned to the right with e.g.
%     an itemize to the left. In this case, specify the number
%     of lines running left of the exhibit.
%
% 4. Optional parameter: alignment. One of:
%     r:		right
%     R:	right, floating
%     i:		inner border
%     I:		inner border, floating
%     l:		left
%     L:		left, floating
%     o:		outer border (default)
%     O:	outer border, floating
%
% 5. Optional parameter: Size as percentage
%     of line width. Over .8 (80%), no wrapfig will be
%     used.
%
%-----------------------------------------------------
% Sample:
%-----------------------------------------------------
%
% \begin{exhibit}[CVP, Price Adjusted]{Price Adjusted CVP}[][.41]
% Nulla malesuada porttitor diam. Donec felis erat, congue non, volutpat at,
% tincidunt tristique, libero. Vivamus viverra fermentum felis. Donec nonummy
% pellentesque ante.  \bigskip
%
% \centerline{\colorbox{white}{\framebox{\includegraphics[width=0.9 \linewidth]{fig/cvp_variable_pricing.pdf}}}}
% \captionof{figure}[LabelInTOC]{FigureCaption}
% \captionsetup{font={footnotesize,it}}
% \vspace{-0.3cm}
% \captionof*{figure}{Source: Bla.}
% \label{fig:Label}
% \bigskip
%
% Donec felis erat, congue non, volutpat at, tincidunt tristique,
% \end{exhibit}
% \ref{exhibit:Price_Adjusted_CVP}
%-----------------------------------------------------
\ifpdf
\definecolor{exhibittitlebackground}{rgb}{.698,.780,.7411}
\definecolor{exhibittitlefont}{rgb}{1.,1,.7}
\definecolor{exhibitbodybackground}{rgb}{.733,.769,.847}
\else
%\Configure{HColor}{exhibittitlebackground}{rgb(.698,.780,.7411)}
%\Configure{HColor}{exhibittitlefont}{rgb(1.,1,.7)}
%\Configure{HColor}{exhibitbodybackground}{rgb(.733,.769,.847)}
\fi

\setlength{\intextsep}{0cm plus1cm minus1cm}

\newcommand{\listexhibitname}{List of Exhibits}
\newlistof{exhibit}{loe}{\listexhibitname}
\ifthenelse{\equal{\doctype}{book}}{
%  \renewcommand{\theexhibit}{\arabic{chapter}.\arabic{exhibit}}
  \renewcommand{\theexhibit}{\arabic{exhibit}}
}{
  \renewcommand{\theexhibit}{\arabic{exhibit}}
}
\makeatletter
\NewDocumentEnvironment{exhibit}{O{} m O{} O{.5} O{o}}{%
\refstepcounter{exhibit}%
\StrSubstitute[0]{#2}{ }{-}[\m@label]%
\label{exhibit:\m@label}%
%
% If first parameter is empty, utilize caption for list of exhibits
%
\ifthenelse{\isempty{#1}{}}{\def\m@tcapt{#2}}{\def\m@tcapt{#1}}%
%
% Create variable for caption
%
\def\m@capt{#2}%
%
%
% Test for default value of alignment
%
\ifthenelse{\isempty{#5}}{\def\m@align{o}}{\def\m@align{#5}}%
%
% Test for width
%
\ifthenelse{\isempty{#4}}{\def\m@width{.5}}{%
  \ifthenelse{\lengthtest{#4 pt  > .973 pt }}{%
    \def\m@width{.973}%
  }{%
    \def\m@width{#4}%
  }%
}%
%
% Add entry to list of exhibits
%
\ifpdf
\addcontentsline{loe}{exhibit}{\protect\numberline{\theexhibit}\m@tcapt}\par%
\fi
%
% Add wrapfig environment, but only if not too wide
%
\wcounte%
\ifpdf%
\ifthenelse{\lengthtest{\m@width pt  < .81 pt }}{%
  \ifthenelse{\isempty{#3}}{%
    \wrapfigure{\m@align}[0pt]{0pt}% Alignment: rRlLiIoO
  }{%
    \wrapfigure[#3]{\m@align}[0pt]{0pt}% Alignment: rRlLiIoO
  }%
}{%
  \center%
}%
\else%
  \center%
\fi%
\tabular{p{\m@width\textwidth}}\toprule%
\ifpdf%
\rowcolor{exhibittitlebackground}\color{exhibittitlefont}\raisebox{5pt}{%
\fi%
\fontsize{11}{13}\selectfont\textsc{Exhibit~\theexhibit: \m@capt}
\ifpdf%
}%
\fi%
\\\midrule %
\ifpdf%
\rowcolor{exhibitbodybackground}%
\fi%
\fontsize{10}{13}\selectfont%
}{%
\\\bottomrule\\\endtabular%
\wcounta%
%
% Finish wrapfig environment only if not too wide
%
\ifpdf%
\ifthenelse{\lengthtest{\m@width pt < .81 pt}}{%
  \endwrapfigure%
}{%
  \endcenter%
}%
\else%
  \endcenter%
\fi%
\vspace*{-\parskip}%
}%
\makeatother




%-----------------------------------------------------
% Epigraph
%-----------------------------------------------------


\definecolor{quotationcolour}{rgb}{0.95,0.95,0.95}
\definecolor{quotationmarkcolour}{rgb}{0,0.4,0.9}

% Double-line for start and end of epigraph.
\newcommand{\epiline}{\hrule \vskip -.2em \hrule}
% Massively humongous opening quotation mark.
\newcommand{\hugequote}{%
  \fontsize{42}{48}\selectfont\color{quotationmarkcolour}\textbf{``}%
  \vskip -.5em%
}

% Make the box around quotations fit snugly to the epilines.
%
% 20161126: mnott
%           taking this out moving to scrivener: It does remove
%           all boxes around graphics / frambeox
\setlength\fboxrule{0pt}
\setlength\fboxsep{0pt}

\newcommand{\epigraphbox}[2]{%
\wcounte%
  \bigskip\ni
  %\vskip -1.5em%
  \colorbox{quotationcolour}{%
    \begin{minipage}{\linewidth}%
    \dimen0\linewidth
    \advance\dimen0 by -3.3em
    \hangindent 1.1em%
    \epiline \vskip 1em \hspace{1.1em}{\hugequote} \vskip -.5em%
    \hspace{1.1em}
    \parbox{\the\dimen0}{
        \parindent 2.2em%
        \setlength\marginparsep{-1.1em}%
        \normalsize\itshape%
        #1}%
    \begin{flushright}\color{quotationmarkcolour}\vskip .5em%
    \footnotesize\textsc{#2}\hspace*{1em} ~\end{flushright}%
    \vskip .5em%
    \epiline%
    \end{minipage}%
  }%
  \bigskip%
\wcounta
}



%-----------------------------------------------------
% Table colors
%-----------------------------------------------------

\definecolor{unfavorable}{rgb}{.949,.804,.804}
\definecolor{favorable}{rgb}{.804,.949,.804}
\definecolor{lc}{rgb}{0.918, 0.918, 0.894}  % light cell
\definecolor{dc}{rgb}{0.757, 0.757, 0.710} % dark cell
\definecolor{tl}{rgb}{0.161, 0.592, 0.404} % top line
\definecolor{ml}{rgb}{0.757, 0.757, 0.710} % mid line
\definecolor{bl}{rgb}{0.600, 0.604, 0.541} % bottom line






%=====================================================
% Massively ugly workaround of hyperrefs issues with equations in glossaries
%=====================================================

%===<exclude> for word count

\ifpdf
  \makeatletter
  \renewcommand{\theHequation}{\@currentHref.\arabic{equation}}
  \gdef\equationgrouping{}
  \makeatother
\fi

%===</exclude>


%=====================================================
% Glossaries
%=====================================================

\usepackage[
nonumberlist, 													% do not show page numbers
acronym,        												% generate acronym listing
toc,                										% show listings as entries in table of contents
section]         												% use section level for toc entries
{glossaries}

%
% We keep the usepackage so that we don't bail out on
% existing \gls... but then for the rest of it, we don't
% even do it if we don't show glossaries
%

%
% Patch Glossaries so that only the first occurrence of a given glossary
% entry is converted to a hyperlink, in order to avoid cluttering.
%
\makeatletter
\ifnum\pdfstrcmp{\showanyglossary}{true}=\z@
%% patch first occurence of "\@gls@link[#1]{#2}{\@glo@text}",
%% as this is the one for \glsused{#2}
\patchcmd{\@gls@}
  {\@gls@link[#1]{#2}{\@glo@text}}
  {\@gls@link[#1,hyper=false]{#2}{\@glo@text}}
  {}{}
\patchcmd{\@glspl@}
  {\@gls@link[#1]{#2}}
  {\@gls@link[#1,hyper=false]{#2}}
  {}{}
\patchcmd{\@Gls@}
  {\@gls@link[#1]{#2}}
  {\@gls@link[#1,hyper=false]{#2}}
  {}{}
\patchcmd{\@Glspl@}
  {\@gls@link[#1]{#2}}
  {\@gls@link[#1,hyper=false]{#2}}
  {}{}
  \patchcmd{\@GLS@}
  {\@gls@link[#1]{#2}{\MakeUppercase{\@glo@text}}}
  {\@gls@link[#1,hyper=false]{#2}{\MakeUppercase{\@glo@text}}}
  {}{}
\fi
\makeatother

%
% Generate a list of symbols
%
\newglossary[slg]{symbolslist}{syi}{syg}{List of Symbols}

\sequals{\showanyglossary}{true}{

%
% Make sure first character of glossary entry name is uppercase
%
\renewcommand{\glsnamefont}[1]{\makefirstuc{#1}}

%
% Remove the dot at the end of glossary descriptions
%
\renewcommand*{\glspostdescription}{}

%
% Activate glossary commands
%
\makeglossaries

%
% Load the glossary definitions the user writes
%
%\input{chapter_a1_acronyms}
%\input{chapter_a2_symbols}
%\input{chapter_a3_glossary}

%
% These commands actually create / update the different
% indices / glossaries
%
%makeindex -s document.ist -t document.alg -o document.acr document.acn
%makeindex -s document.ist -t document.glg -o document.gls document.glo
%makeindex -s document.ist -t document.slg -o document.syi document.syg
%makeindex document

}{} % <= showanyglossary

%=====================================================
% Index
%=====================================================

\usepackage{makeidx}

\makeatletter
\renewenvironment{theindex}
   {%
    \@mkboth{\MakeUppercase\indexname}%
            {\MakeUppercase\indexname}%
    \thispagestyle{plain}\parindent\z@
    \parskip\z@ \@plus .3\p@\relax
    \columnseprule \z@
    \columnsep 35\p@
    \let\item\@idxitem}
   {}
\makeatother

%=====================================================
% Graphics
%=====================================================
\usepackage{graphicx}
\ifpdf
  \graphicspath{{pdf/}}
  \pdfcompresslevel=9
  \DeclareGraphicsExtensions{.pdf}
  \DeclareGraphicsRule{.pdf}{pdf}{.pdf}{}
\else
  \graphicspath{{eps/}}
  \DeclareGraphicsExtensions{.eps}
  \DeclareGraphicsRule{.eps}{eps}{.eps}{}
\fi


%=====================================================
% Media
%=====================================================

%\renewcommand{\video}[6]{% file xpos ypos width height controls
%  \vspace{#3}\hspace{#2}{\pdfannot width #4 height #5 depth 0cm {%
%   /Subtype /Movie
%   /Movie  << /F (#1) >>
%   /A << /ShowControls #6 /Rate 1 >>
%   }}}
%\fi

\usepackage{templates/sty/easymovie/easymovie}


%=====================================================
% Equations
%=====================================================

\usepackage[fleqn,tbtags]{mathtools}	% Mathematical Processing
\usepackage{amssymb}									% Scientific Symbols
\usepackage{latexsym}									%	Scientific Symbols
\mathtoolsset{showonlyrefs}						% Label only (eqref) referenced Equations
%\everymath{\rm}											% Default to roman style

\usepackage[customcolors]{hf-tikz}				% Highlight Formulas
\usetikzlibrary{calc}
\tikzstyle{every picture}+=[remember picture]
\hfsetfillcolor{blue!10}
\hfsetbordercolor{blue}

% Circled characters

\newcommand*\circled[1]{\tikz[baseline=(char.base)]{
            \node[shape=circle,draw,inner sep=2pt] (char) {#1};}}

%=====================================================
% Font Settings
%=====================================================

\usepackage{microtype}								% More precise typography
\usepackage{fix-cm}										% Permit arbitrary font sizes
\usepackage[right]{eurosym}		   		  % Euro Symbol
\usepackage{pifont}                   % Dingbats

%
% Specify fonts for Captions, Sectionts, etc.
%
\renewcommand*\captionlabelfont{\bfseries}
\renewcommand*\captionsize{\itshape}

\makeatletter
\renewcommand{\section}{\@startsection{section}{1}{\z@}%
    {-2.2ex \@plus-1ex \@minus -.2ex}{1.3ex \@plus.2ex}%
    {\reset@font\large\bfseries}}
\renewcommand{\subsection}{\@startsection{subsection}{2}{\z@}%
    {-1.5ex \@plus -1ex \@minus-.2ex}{0.8ex \@plus.2ex}%
    {\reset@font\normalsize\bfseries}}
\renewcommand{\subsubsection}{\@startsection{subsubsection}{3}{\z@}%
     {-1.2ex\@plus -1ex \@minus -.2ex}{0.5ex \@plus .2ex}%
     {\reset@font\normalsize}}
 \renewcommand{\paragraph}{\@startsection{paragraph}{4}{0mm}%
  {1ex \@plus1ex \@minus.2ex}%
  {-1em}%
  {\normalfont\normalsize\it}}
 \renewcommand{\subparagraph}{\@startsection{subparagraph}{5}{\parindent}%
  {2.0ex \@plus1ex \@minus .2ex}%
  {-1em}%
  {\normalfont\normalsize\it}}
\makeatother

%
% And for Glossaries:
%
\makeatletter
\newcommand{\glosection}{\@startsection{section}{1}{\z@}%
    {-2.2ex \@plus-1ex \@minus -.2ex}{1.3ex \@plus.2ex}%
    {\reset@font\normalfont\sc}}
\newcommand{\glosubsection}{\@startsection{subsection}{2}{\z@}%
    {-1.5ex \@plus -1ex \@minus-.2ex}{0.8ex \@plus.2ex}%
    {\reset@font\normalsize\itshape}}
\makeatother

%
% Itemizes
%
\renewcommand{\labelitemi}{$\triangleright$}
\renewcommand*\descriptionlabel[1]{\hspace\labelsep
                                \normalfont\itshape #1}

%
% Save Default Space above Itemize etc., then set it to 0
%
% Not resetting, need further investigation.
%
% \newlength{\oldabovedisplayskip}
% \setlength{\oldabovedisplayskip}{\abovedisplayskip}
% %\setlength{\abovedisplayskip}{0pt}
% \expandafter\def\expandafter\normalsize\expandafter{%
% \normalsize\setlength\oldabovedisplayskip{\abovedisplayskip}}
% \expandafter\def\expandafter\normalsize\expandafter{%
% \normalsize\setlength\abovedisplayskip{0pt}}
% \expandafter\def\expandafter\normalsize\expandafter{%
% \normalsize\setlength\abovedisplayskip{\oldabovedisplayskip}}


%=====================================================
% Table of... / Listofs
%=====================================================

\usepackage[titles]{tocloft}						% New Listofs...
\setcounter{secnumdepth}{10}						% Section numbers down to level 10
\setcounter{tocdepth}{1}								% TOC content down to level 3

%
% Give some more room for page numbers
%
\makeatletter
\renewcommand{\@pnumwidth}{3em}
\renewcommand{\@tocrmarg}{4em}
\makeatother


%=====================================================
% Show Page Frames
%=====================================================

%\usepackage{showframe}


%=====================================================
% Pandoc Stuff
%=====================================================

%\providecommand{\tightlist}{\setlength{\itemsep}{4pt}}

\providecommand{\tightlist}{%
  \setlength{\itemsep}{5pt}\setlength{\parskip}{5pt}}

\setlength{\abovedisplayskip}{0pt}
\setlength{\belowdisplayskip}{0pt}

